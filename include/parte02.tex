\section{Produto escalar}

\begin{frame}{Medida angular}
    \begin{itemize}
        \item Sejam \((O,P)\) e \((O,Q)\) representantes dos vetores não nulos \(\vec{u}\) e \(\vec{v}\)
        \item Note que esses segmentos orientados possuem a mesma \textit{origem}
        \item Chama-se \textbf{medida angular} entre \(\vec{u}\) e \(\vec{v}\), indicado por \(ang(\vec{u},\vec{v})\), o ângulo \(\theta=P\hat{O}Q\)
        \item O ângulo \(\theta\) é o \textit{menor ângulo positivo} entre os segmentos, ou seja, \( 0 \leq \theta \leq \pi\)
    \end{itemize}

    \begin{center}
        \begin{tikzpicture}[scale=4]
            \coordinate (O) at (0,0);
            \coordinate (P) at (1,0);
            \coordinate (Q) at (1,1);

            \node [below left] at (O) {O};
            \node [above right] at (P) {P};
            \node [below right] at (Q) {Q};

            \draw[thick] (O) -- (P);
            \draw[thick] (O) -- (Q);
            \draw[thick, red] ($0.5*(P)$) arc (0:45:0.5) node[right,pos=0.5] {$\theta$};
        \end{tikzpicture}
    \end{center}

\end{frame}

\begin{frame}{Produto escalar}
    \textbf{Produto escalar} dos vetores \(\vec{u}\) e \(\vec{v}\), indicado por \(\vec{u}\cdot\vec{v}\), é definido como:
    \[
        \vec{u}\cdot\vec{v}=
        \begin{cases}
            0 &\text{ se } \vec{u} \text{ ou } \vec{v} = \vec{0} \\
            ||\vec{u}||~||\vec{v}|| ~\cos{\theta} & \text{ se } \vec{u} \text{ e } \vec{v} \text{ são não nulos}\\
        \end{cases}
    \]
    onde \(\theta = ang(\vec{u},\vec{v})\)

    Note que:
    \begin{itemize}
        \item \(||\vec{u}|| = \sqrt{\vec{u}\cdot\vec{u}}\)
        \item \(\vec{u} \perp \vec{v} \Leftrightarrow \vec{u}\cdot\vec{v}=0\)
        \item \(\vec{u}\cdot\vec{v} = \vec{u}\cdot\vec{w}\) não significa que \(\vec{v}=\vec{w}\) (não dá para ''cortar'')
        \item \(\vec{u}\cdot\vec{v}=0\) não significa que \(\vec{u}\) ou \(\vec{v}\) seja nulo
    \end{itemize}
\end{frame}

\begin{frame}{Propriedades}

    Quaisquer que sejam os vetores \(\vec{u}\), \(\vec{v}\) e \(\vec{w}\) e qualquer que seja o escalar \(\lambda\), valem as propriedades:
    \begin{itemize}
        \item \(\vec{u}\cdot (\vec{v}+\vec{w})=\vec{u}\cdot\vec{v}+\vec{u}\cdot\vec{w}\)
        \item \(\vec{u}\cdot (\lambda\vec{v})=(\lambda\vec{u})\cdot\vec{v}=\lambda (\vec{u}\cdot\vec{v})\)
        \item \(\vec{u}\cdot\vec{v}=\vec{v}\cdot\vec{u}\)
        \item \(||\vec{u}+\vec{v}||^2=||\vec{u}||^2+2\vec{u}\cdot\vec{v}+||\vec{v}||^2\)
        \item \(\vec{u}\cdot\vec{v}\) \textbf{não depende da base usada}
        \item se \(\vec{u}\), \(\vec{v}\) e \(\vec{w}\) são vetores não nulos e ortogonais, então \((\vec{u},\vec{v},\vec{w})\) é LI

    \end{itemize}

\end{frame}

\begin{frame}{}
    Seja uma base ortonormal \(E=(\vec{e_1},\vec{e_2},\vec{e_3})\) e \(\vec{u}=(a_1,a_2,a_3)_E\) e \(\vec{v}=(b_1,b_2,b_3)_E\)
    \begin{itemize}
        \item \(\vec{e_i}\cdot\vec{e_j}=\delta_{ij}\), onde
            \[
                \delta_ij=%
                \begin{cases}
                    0 &\text{ se } i \neq j \\
                    1 &\text{ se } i = j\\
                \end{cases}
            \]
        \item \(\vec{u}\cdot\vec{v}=a_1 b_1 +a_2 b_2 + a_3 b_3\)
        \item \(a_i = \vec{u}\cdot\vec{e_i}\)
    \end{itemize}

\end{frame}

\begin{frame}[c]{Exercício 9-6}
    Os lados do triângulo equilátero \(ABC\) têm medida 2. Calcule
    \[
        \vec{AB}\cdot\vec{BC}+\vec{BC}\cdot\vec{CA}+\vec{CA}\cdot\vec{AB}
    \]
\end{frame}

\begin{frame}{Exercício 9-6}
    ''\textit{A soma dos ângulos internos de um triângulo é igual a \SI{180}{\degree}}''
    \begin{center}
        \begin{tikzpicture}[scale=2]
            \draw[thick] (0,0) coordinate[label=below:A] (A) --
                (1,0) coordinate[label=below:B] (B) --
                ($0.5*(1,0)+0.5*sqrt(3)*(0,1)$) coordinate[label=above:C] (C) --
                cycle;
            \draw[red] (0.15,0) arc (0:60:0.15);
            \node[red,right] at (0.1,0.15) {$60^\circ$};
        \end{tikzpicture}
    \end{center}
    \begin{itemize}
        \item \(
            \vec{AB}\cdot\vec{BC}=
            2\cdot 2 \cos{(\SI{180}{\degree}-\SI{60}{\degree})}=
            4\cdot\cos{\SI{120}{\degree}}=-2
            \)
        \item \(
            \vec{BC}\cdot\vec{CA}=
            2\cdot 2 \cos{(\SI{180}{\degree}-\SI{60}{\degree})}=
            4\cdot\cos{\SI{120}{\degree}}=-2
            \)
        \item \(
            \vec{CA}\cdot\vec{AB}=
            2\cdot 2 \cos{(\SI{180}{\degree}-\SI{60}{\degree})}=
            4\cdot\cos{\SI{60}{\degree}}=-2
            \)
        \item \(\vec{AB}\cdot\vec{BC} + \vec{BC}\cdot\vec{CA} +
            \vec{CA}\cdot\vec{AB}=-2-2-2=-6\)
    \end{itemize}
\end{frame}

\begin{frame}[c]{Exercício 9-5}
    Sendo \(ABCD\) um tetraedro regular de aresta unitária, calcule \(\vec{AB}\cdot\vec{DA}\)
\end{frame}

\begin{frame}{Exercício 9-5}
    \begin{itemize}
        \item \textit{O que é um tetraedro regular?}
        \item \textit{Google}:
            ''Quando um tetraedro possui as quatro faces e os quatro
            bicos iguais (também chamados ângulos poliédricos), temos um
            tetraedro regular.''
        \item Portanto, \(\vec{AB}\) e \(\vec{AD}\) são arestas de um triângulo equilátero:
            % \begin{center}
            %     \begin{tikzpicture}[scale=2]
            %         \draw[thick] (0,0) coordinate[label=below:A] (A) --
            %             (1,0) coordinate[label=below:B] (B) --
            %             ($0.5*(1,0)+0.5*sqrt(3)*(0,1)$) coordinate[label=above:D] (D) --
            %             cycle;
            %         \draw[red] (0.15,0) arc (0:60:0.15);
            %         \node[red,right] at (0.1,0.15) {$60^\circ$};
            %     \end{tikzpicture}
            % \end{center}
        % \item Se o ângulo entre \(\vec{AB}\) e \(\vec{AD}\) é \SI{60}{\degree}, então o ângulo entre \(\vec{AB}\) e \(\vec{DA}\) é
            % \(\SI{180}{\degree}-\SI{60}{\degree}=\SI{120}{\degree}\)
        \item Como a aresta é unitária, temos que
            \(\vec{AB}\cdot\vec{DA}=1 \cdot 1 \cdot \cos{\SI{120}{\degree}}=-1/2\)
    \end{itemize}
\end{frame}

\begin{frame}[c]{Exercício 9-7}
    São dados os números reais positivos \(a\) e \(b\), e o vetor \(\vec{u}\), de norma \(a\).
    Dentre os vetores de norma \(b\), qual é o que torna máximo o produto escalar \(\vec{u}\cdot\vec{v}\)?
    E mínimo? Quais são esses valores máximo e mínimo?
\end{frame}

\begin{frame}{Exercício 9-7}
    \begin{itemize}
        \item \(
            \vec{u}\cdot\vec{v}=||\vec{u}||\cdot ||\vec{v}||\cos{\theta}=
            ab\cos{\theta}
            \)
        \item Mas \(-1 \leq \cos{\theta} \leq 1\), onde \(
            \cos{\theta}=-1 \implies\theta=\SI{180}{\degree}\) e \(
            \cos{\theta}=1 \implies\theta=\SI{0}{\degree}\)
        \item \(\theta=\SI{0}{\degree} \implies \) vetores com mesmo sentido e direção
        \item \(\theta=\SI{180}{\degree} \implies \) vetores com mesma direção e sentidos opostos
        \item Assim, o vetor \(\vec{v}\) que dá o maior produto escalar é um vetor \textbf{paralelo} a \(\vec{u}\) com módulo \(b\), ou seja,
            \[
                \vec{v}=\frac{b}{a}\vec{u}
            \]
            de acordo com a definição de produto de um escalar por vetor e \(
            \vec{u}\cdot\vec{v}=ab\)

        \item Analogamente, o vetor \(\vec{v}\) que dá o menor produto escalar é um vetor \textbf{antiparalelo} a \(\vec{u}\) com módulo \(b\), ou seja,
            \[
                \vec{v}=-\frac{b}{a}\vec{u}
            \]
            de acordo com a definição de produto de um escalar por vetor e \(
            \vec{u}\cdot\vec{v}=-ab\)
    \end{itemize}

\end{frame}

\begin{frame}[c]{Exercício 9-13}
    \begin{enumerate}[(a)]
        \item Obtenha os valores de norma \(3\sqrt{3}\) que são ortogonais a \(\vec{u}=(2,3,-1)\) e a
            \(\vec{v}=(2,-4,6)\)
        \item Qual dos vetores obtidos no item (a) forma ângulo agudo com \((1,0,0)\)
    \end{enumerate}
\end{frame}

\begin{frame}{Exercício 9-13}
    \begin{itemize}
        \item Considere que os vetores procurados têm a forma \(\vec{w}=(x,y,z)\)
        \item Assim, temos o sistema de equações:
            \[
                \begin{cases}
                    2x+3y-z&=0 \\ 2x-4y+6z&=0 \\ \sqrt{x^2+y^2+z^2}&=3\sqrt{3}
                \end{cases}
            \]
            onde a primeira e segunda equação saem da condição de ortogonalidade e a terceira do módulo (norma) de \(\vec{w}\)
        \item ''Resolvendo'' as duas primeiras equações, temos que:
            \[
                \begin{cases}
                    \alert{2x}+3y-z&=0 \\ \alert{2x}-4y+6z&=0
                \end{cases}
                \implies 4y-6z=-3y+z \implies 7y=7z \implies y=z
            \]
        \item Além disso, temos que \(x=-y\)
    \end{itemize}

\end{frame}

\begin{frame}{}
    \begin{itemize}
        \item Como \(x=-y\) e \(z=y\), podemos resolver a terceira equação:
            \[
                \sqrt{x^2+y^2+z^2}=3\sqrt{3} \implies
                \sqrt{(-y)^2+y^2+y^2}=3\sqrt{3} \implies
                \sqrt{3y^2}=3\sqrt{3}
            \]
            \[
                \sqrt{y^2}=3 \implies y=\pm 3
            \]
        \item Assim, temos que a resposta de \alert{a)} é \(\vec{w}=(-3,3,3)\) ou \(\vec{w}=(3,-3,-3)\)
        \item b) Um ângulo agudo é um ângulo menor que \SI{90}{\degree} e \(\cos{\theta} > 0\) quando \( 0 \leq \theta < \SI{90}{\degree}\)
        \item Como \((x,y,z) \cdot (1,0,0)=x\), então somente \((3,-3,-3)\) faz um ângulo agudo com \((1,0,0)\)
    \end{itemize}
\end{frame}

\begin{frame}[c]{Exercício 9-18}
    Descreva o conjunto de todos os vetores \(\vec{w}\) ortogonais a \(\vec{v}=(2,1,2)\) tais que
    \(\vec{u}=(1,1,-1)\) seja combinação linear de \(\vec{v}\), \(\vec{w}\)
\end{frame}

\begin{frame}{Exercício 9-18}
    \begin{itemize}
        \item Considere que \(\vec{w}=(x,y,z)\)
        \item Para \(\vec{w}\) ser ortogonal a \(\vec{v}=(2,1,2)\) temos que
            \[
                \vec{w}\cdot \vec{v}=0 \implies 2x+y+2z=0
            \]
        \item Para \(\vec{u}\) ser combinação linear de \(\vec{v}\) e \(\vec{w}\), deve existir \(a\) e \(b\) tal que
            \[
                (1,1,-1)=a(2,1,2)+b(x,y,z) \implies
                \begin{cases}
                    2a+bx &= 1 \\ a+by&=1 \\ 2a+bz&=-1
                \end{cases}
            \]
        \item Desse sistema temos que
            \[
                2(2a+bx)+(a+by)+2(2a+bz)=2+1-2=1 \implies 9a + b(2x+y+2z)=1 \implies a=\frac{1}{9}
            \]
    \end{itemize}

\end{frame}

\begin{frame}{}
    \begin{itemize}
        \item Substituindo \(a=1/9\) em
            \[
                \begin{cases}
                    2a+bx &= 1 \\ a+by&=1 \\ 2a+bz&=-1
                \end{cases}
            \]
        \item[]
            e isolando \(x\), \(y\) e \(z\), temos que
            \[
                \begin{cases}
                    x&=\frac{7}{9b} \\ y&=\frac{8}{9b} \\ z&=-\frac{11}{9b}
                \end{cases}
            \]
        \item Ou seja,
            \[
                \vec{w}=\frac{1}{9b}(7,8,-11)=\lambda (7,8,-11)
            \]

    \end{itemize}
\end{frame}

