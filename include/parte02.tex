\begin{frame}{Medida angular}
    \begin{itemize}
        \item Sejam \((O,P)\) e \((O,Q)\) representantes dos vetores não nulos \(\vec{u}\) e \(\vec{v}\)
        \item Note que esses segmentos orientados possuem a mesma \textit{origem}
        \item Chama-se \textbf{medida angular} entre \(\vec{u}\) e \(\vec{v}\), indicado por \(ang(\vec{u},\vec{v})\), o ângulo \(\theta=P\hat{O}Q\)
        \item O ângulo \(\theta\) é o \textit{menor ângulo positivo} entre os segmentos, ou seja, \( 0 \leq \theta \leq \pi\)
    \end{itemize}

    \begin{center}
        \begin{tikzpicture}[scale=4]
            \coordinate (O) at (0,0);
            \coordinate (P) at (1,0);
            \coordinate (Q) at (1,1);

            \node [below left] at (O) {O};
            \node [above right] at (P) {P};
            \node [below right] at (Q) {Q};

            \draw[thick] (O) -- (P);
            \draw[thick] (O) -- (Q);
            \draw[thick, red] ($0.5*(P)$) arc (0:45:0.5) node[right,pos=0.5] {$\theta$};
        \end{tikzpicture}
    \end{center}

\end{frame}

\begin{frame}{Produto escalar}
    \textbf{Produto escalar} dos vetores \(\vec{u}\) e \(\vec{v}\), indicado por \(\vec{u}\cdot\vec{v}\), é definido como:
    \[
        \vec{u}\cdot\vec{v}=
        \begin{cases}
            0 &\text{ se } \vec{u} \text{ ou } \vec{v} = \vec{0} \\
            ||\vec{u}||~||\vec{v}|| ~\cos{\theta} & \text{ se } \vec{u} \text{ e } \vec{v} \text{ são não nulos}\\
        \end{cases}
    \]

    onde \(\theta = ang(\vec{u},\vec{v})\)

    Note que:
    \begin{itemize}
        \item \(||\vec{u}|| = \sqrt{\vec{u}\cdot\vec{u}}\)
        \item \(\vec{u} \perp \vec{v} \Leftrightarrow \vec{u}\cdot\vec{v}=0\)
        \item \(\vec{u}\cdot\vec{v} = \vec{u}\cdot\vec{w}\) não significa que \(\vec{v}=\vec{w}\) (não dá para ''cortar'')
        \item \(\vec{u}\cdot\vec{v}=0\) não significa que \(\vec{u}\) ou \(\vec{v}\) seja nulo
    \end{itemize}
\end{frame}

\begin{frame}{Propriedades}

    Quaisquer que sejam os vetores \(\vec{u}\), \(\vec{v}\) e \(\vec{w}\) e qualquer que seja o escalar \(\lambda\), valem as propriedades:
    \begin{itemize}
        \item \(\vec{u}\cdot (\vec{v}+\vec{w})=\vec{u}\cdot\vec{v}+\vec{u}\cdot\vec{w}\)
        \item \(\vec{u}\cdot (\lambda\vec{v})=(\lambda\vec{u})\cdot\vec{v}=\lambda (\vec{u}\cdot\vec{v})\)
        \item \(\vec{u}\cdot\vec{v}=\vec{v}\cdot\vec{u}\)
        \item \(||\vec{u}+\vec{v}||^2=||\vec{u}||^2+2\vec{u}\cdot\vec{v}+||\vec{v}||^2\)
        \item \(\vec{u}\cdot\vec{v}\) \textbf{não depende da base usada}
        \item se \(\vec{u}\), \(\vec{v}\) e \(\vec{w}\) são vetores não nulos e ortogonais, então \((\vec{u},\vec{v},\vec{w})\) é LI

    \end{itemize}

\end{frame}

\begin{frame}{}
    Seja uma base ortonormal \(E=(\vec{e_1},\vec{e_2},\vec{e_3})\) e \(\vec{u}=(a_1,a_2,a_3)_E\) e \(\vec{v}=(b_1,b_2,b_3)_E\)
    \begin{itemize}
        \item \(\vec{e_i}\cdot\vec{e_j}=\delta_{ij}\), onde
            \[
                \delta_ij=%
                \begin{cases}
                    0 &\text{ se } i \neq j \\
                    1 &\text{ se } i = j\\
                \end{cases}
            \]
        \item \(\vec{u}\cdot\vec{v}=a_1 b_1 +a_2 b_2 + a_3 b_3\)
        \item \(a_i = \vec{u}\cdot\vec{e_i}\)
    \end{itemize}

\end{frame}
