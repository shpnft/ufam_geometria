\begin{frame}{Atividade valendo 70\% da \(N_3\)}
    \begin{itemize}
        \item Grupo de no máximo 3 alunos
        \item Conteúdo
            \begin{enumerate}
                \item Elipse: definição, equação reduzida e o que for necessário para a \textbf{resolução do exercício 22-7}
                \item Hipérbole: definição, equação reduzida e o que for necessário para a \textbf{resolução do exercício 22-21}
                \item Parábola: definição, equação reduzida e o que for necessário para a \textbf{resolução do exercício 22-32}
            \end{enumerate}
        \item Apresentação em vídeo
            \begin{itemize}
                \item A imagem dos alunos não precisa aparecer no vídeo:
                    apenas a voz de um ou mais integrantes do grupo
                \item Exemplo totalmente amador: o aluno faz a apresentação enquanto segura o celular 
                    gravando a tela do notebook/folha do papel/quadro/cartolina(?)
                \item Não é necessário entregar trabalho escrito
            \end{itemize}
        \item Data de entrega: 
            \begin{itemize}
                \item presencialmente (copiar para meu computador): até 17/2
                \item via e-mail (link para o Youtube, vídeo não listado ou privado): 24/2
            \end{itemize}

    \end{itemize}
\end{frame}

\begin{frame}{Equação \textit{de} reta}
    \begin{itemize}
        \item Qualquer vetor não nulo paralelo a uma reta chama-se \textbf{vetor diretor}
            dessa reta
        \item Sejam \(\vec{u}\) \textbf{um} vetor diretor de uma reta \(r\) e \(A\)
            um ponto pertencente a \(r\)
        \item Um ponto \(X\) pertence a \(r\) se, e somente se, existe um número real \(\lambda\) tal que:
            \[
                X=A+\lambda \vec{u}
            \]
            chamamos essa equação de \textbf{equação da reta \(r\) na forma vetorial}
        \item Observações:
            \begin{itemize}
                \item todo ponto \(X\) de \(r\) está associado um escalar \(\lambda\)
                \item há infinitas equações para uma mesma reta
                \item podemos escrever \(r: X=A+\lambda \vec{u}\) para indicar que estamos 
                    tratando da equação da reta \(r\)
                \item a forma vetorial da equação da reta \textit{não precisa de sistema de coordenadas}
            \end{itemize}
    \end{itemize}
\end{frame}

\begin{frame}
    \begin{itemize}
        \item Se \(X=(x,y,z)\), \(A=(x_0,y_0,z_0)\) e \(\vec{u}=(a,b,c)\), 
            podemos reescrever \(X=A+\lambda \vec{u}\) na forma
            \[
                \begin{cases}
                    x = x_0 + \lambda a \\
                    y = y_0 + \lambda b \\
                    z = z_0 + \lambda c
                \end{cases}
            \]
        chamamos essas equações de \textbf{equações da reta \(r\) na forma paramétrica}
    \item Observações:
        \begin{itemize}
            \item as formas vetorial e paramétrica são equivalentes
            \item temos um sistema de equações com infinitas soluções (4 incógnitas e 3 equações): 
                o conjunto de pontos de \(r\) está associado ao conjunto de soluções desse sistema
            \item da mesma forma, podemos associar um \alert{sistema de equações} a uma reta
            \item a forma paramétrica das equações da reta \textit{precisa de sistema de coordenadas}
        \end{itemize}
    \end{itemize}
\end{frame}

\begin{frame}
    \begin{itemize}
        \item Finalmente, se \(a \neq 0\), \(b \neq 0 \) e \(c \neq 0\), podemos escrever
            \[u
                \frac{x-x_0}{a} = \frac{y-y_0}{b} = \frac{z-z_0}{c}
            \]
            chamamos essa equação de \textbf{equação da reta \(r\) na forma simétrica}
        \item Observações:
            \begin{itemize}
                \item a forma simétrica da equação da reta \textit{precisa de sistema de coordenadas}
                \item lembre-se que há infinitas equações para uma mesma reta e
                    cada uma delas têm formas vetorial, paramétrica e simétrica diferentes
                \item as formas da equação da reta são rígidas e devem ser respeitadas, ou seja:
                    \[
                        \begin{cases}
                            2x=1-3\lambda \\ -y = 2+\lambda \\ z=4+2\lambda
                        \end{cases}
                        \qquad \text{ e } \qquad
                        \frac{2x-7}{3} = \frac{y}{2} = \frac{3-z}{4}
                    \]
                    são equações da reta mas não estão em nenhuma das 3 formas discutidas
            \end{itemize}
    \end{itemize}

\end{frame}

\begin{frame}{Cálculo I}
    Sendo
    \[
        \begin{cases}
            x = x_0 + \lambda a \\
            y = y_0 + \lambda b \\
            z = z_0 + \lambda c
        \end{cases}
    \]

    Note que se \(c=0\) e \(z_0 = 0\), temos que \(z=0\) e \textit{podemos escrever}:
    \[
        \frac{x-x_0}{a} = \frac{y-y_0}{b} \implies y = mx + b'
    \]
    uma forma bastante conhecida para quem estuda Cálculo I
\end{frame}


\begin{frame}{Equação do plano}
    \begin{itemize}
        \item Se \(\vec{u}\) e \(\vec{v}\) são LI e paralelos a um plano \(\pi\),
            o par \((\vec{u},\vec{v})\) é chamado \textbf{par de vetores diretores}
            de \(\pi\) ou simplesmente \textbf{vetores diretores} de \(\pi\)
        \item Se \(A\) é um ponto do plano \(\pi\), \(X\) pertence a \(\pi\) se,
            e somente se, \((\vec{u},\vec{v},\vec{AX})\) é LD e, como vimos anteriormente,
            \[
                \vec{AX} = \lambda \vec{u} + \mu \vec{v} \implies X = A +\lambda \vec{u} + \mu \vec{v}
            \]
            chamamos a segunda equação de \textbf{equação do plano \(\pi\) na forma vetorial}
        \item Observações:
            \begin{itemize}
                \item todo ponto \(X\) de \(\pi\) está associado aos escalares \(\lambda\) e \(\mu\)
                \item há infinitas equações para um mesmo plano
                \item podemos escrever \(\pi: X=A+\lambda \vec{u} +\mu \vec{v}\) para indicar que estamos 
                    tratando da equação do plano \(\pi\)
                \item a forma vetorial da equação do plano \textit{não precisa de sistema de coordenadas}
            \end{itemize}
    \end{itemize}

    \begin{tcolorbox}[colback=magenta!20]
        Descreva a forma paramétrica das equações do plano sabendo que \(X=(x,y,z)\),
        \(A=(x_0,y_0,z_0)\), \(\vec{u}=(r,s,t)\) e \(\vec{v}=(m,n,p)\)
    \end{tcolorbox}
\end{frame}

\begin{frame}{Equação do plano na forma geral}
    \begin{itemize}
        \item Vimos que \(A\) e \(X\) pertencem a \(\pi\) se, e somente se, 
            \((\vec{u},\vec{v},\vec{AX})\) é LD, ou seja:
            \[
                \begin{vmatrix}
                    x-x_0 & y-y_0 & z-z_0 \\
                    r & s & t \\
                    m & n & p
                \end{vmatrix}
                =0
            \]
        \item Resolvendo o determinante, temos
            \[
                \begin{vmatrix}
                    s & t \\ n & p
                \end{vmatrix}
                (x-x_0) -
                \begin{vmatrix}
                    r & t \\ m & p
                \end{vmatrix}
                (y-y_0) +
                \begin{vmatrix}
                    r & s \\ m & n
                \end{vmatrix}
                (z-z_0) =0 \implies ax+ by +cz + d=0
            \]
            % onde
            % \[
            %     \begin{vmatrix}
            %         s & t \\ n & p
            %     \end{vmatrix}
            %     =a \quad
            %     \begin{vmatrix}
            %         r & t \\ m & p
            %     \end{vmatrix}
            %     =b \quad
            %     \begin{vmatrix}
            %         r & s \\ m & n
            %     \end{vmatrix}
            %     =c \quad
            %     -ax_0- by_0 -cz_0 = d
            % \]
        \item a equação \(ax+ by +cz + d=0\) é chamada \textbf{equação do plano na forma geral}
        \item \textit{Toda equação de primeiro grau é equação de um plano}
    \end{itemize}
\end{frame}

\begin{frame}{Exercício 14-1}
    \begin{minipage}{\textwidth}
        Estudando Geometria Analítica em uma noite de sábado, Amanda resolveu
        vários exercícios que pediam equações de reta. Relacionamos a seguir as
        respostas dela e as do livro. Quais exercícios Amanda acertou?
    \end{minipage}
    \begin{itemize}
        \item Exercício A: \(\color{blue} X=(1,2,1)+\lambda (-1,2,1) \quad \color{red} X=(1,2,1) + \lambda (-1/2,1,1/2)\)
        \item Exercício B: \(\color{blue} X=(1/3,-1/3,2/3)+\lambda (-1,1,-1) \quad \color{red} X=(1,-1,2) + \lambda (-1,1,-1)\)
        \item Exercício B: \(\color{blue} X=(1,1,0)+\lambda (1,0,-1/2) \quad \color{red} X=(0,1,1/2) + \lambda (-2,0,1)\)
    \end{itemize}
\end{frame}

\begin{frame}{Exercício 14-20}
    \begin{minipage}{\textwidth}
        Em uma tarde chuvosa, Cecília resolveu vários exercícios de Geometria
        Analítica que pediam equações de planos e foi ao final do livro conferir
        as respostas. Relacionamos a seguir, para cada exercício, as duas respostas:
        a de Cecília e a do livro. Quais exercícios ela acertou?
    \end{minipage}
    \begin{itemize}
        \item Exercício 1: \\ \(
            \color{blue} X=(1,2,1) + \lambda (1,-1,2) + \mu (-1/2,2/3,-1) ~~ 
            \color{red} X=(1,2,1)+\lambda (-1,1,-2)+\mu(-3,4,-6)
            \)
        \item Exercício 2: \\ \(
            \color{blue} X=(1,1,1) + \lambda (2,3,-1) + \mu (-1,1,1) ~~
            \color{red} X=(1,6,2) + \lambda (-1,1,1) +\mu (2,3,-1)
            \)
        \item Exercício 3: \\ \(
            \color{blue} X=(0,0,0)+\lambda (1,1,0) +\mu(0,1,0) ~~
            \color{red} X=(1,1,0) + \lambda (1,2,1) +\mu (0,-1,1)
            \)
        \item Exercício 4: \\ \(
            \color{blue} X=(2,1,3)+\lambda (1,1,-1) + \mu (1,0,1) ~~
            \color{red} X=(0,1,1)+\lambda(1,3,-5)+\mu(1,-1,3)
            \)
    \end{itemize}
\end{frame}

\begin{frame}{Exercício 14-2}
    \begin{minipage}{\textwidth}
        \begin{enumerate}[(a)]
            \item Sejam \(B=(-5,2,3)\) e \(C=(4,-7,6)\). Escreva equações nas
                formas vetorial, paramétrica e simétrica para a reta \(BC\).
                Verifique se \(D=(3,1,4)\) pertence a essa reta.
            \item Dados \(A=(1,2,3)\) e \(\vec{u}=(3,2,1)\), escreva equações
                da reta que contém \(A\) e é paralela a \(\vec{u}\), nas formas
                vetorial, paramétrica e simétrica. Supondo que o sistema de coordenadas
                seja ortogonal, obtenha dois vetores diretores unitários dessa reta.
        \end{enumerate}
    \end{minipage}
\end{frame}

\begin{frame}{Exercício 14-21}
    \begin{minipage}{\textwidth}
        Escreva uma equação vetorial e equações paramétricas do plano \(\pi\),
        utilizando as informações dadas em cada caso.
        \begin{enumerate}[(a)]
            \item \(\pi\) contém \(A=(1,2,1)\) e é paralelo aos vetores \(\vec{u}=(1,1,0)\) e
                \(\vec{v}=(2,3,-1)\)
            \item \(\pi\) contém \(A=(1,1,0)\) e \(B=(1,-1,-1)\) e é paralelo ao vetor
                \(\vec{v}=(2,1,0)\)
            \item \(\pi\) contém \(A=(1,0,1)\) e \(B=(0,1,-1)\) e é paralelo ao segmento de
                extremidades \(C=(1,2,1)\) e \(D=(0,1,0)\)
            \item \(\pi\) contém os pontos \(A=(1,0,1)\), \(B=(2,1,-1)\) e \(C=(1,-1,0)\)
            \item \(\pi\) contém os pontos \(A=(1,0,2)\), \(B=(-1,1,3)\) e \(C=(3,-1,1)\)
        \end{enumerate}
    \end{minipage}
\end{frame}

\begin{frame}{Exercício 14-29}
    \begin{minipage}{\textwidth}
        Obtenha uma equação geral do plano \(\pi\) em cada caso
        \begin{enumerate}[(a)]
            \item \(\pi\) contém \(A=(1,1,0)\) e \(B=(1,-1,-1)\) e é paralelo a \(\vec{u}=(2,1,0)\)
            \item \(\pi\) contém \(A=(1,0,1)\) e \(B=(0,1,-1)\) e é paralelo a \(CD\), sendo
                \(C=(1,2,1)\) e \(D=(0,1,0)\)
            \item \(\pi\) contém \(A=(1,0,1)\), \(B=(2,1,-1)\) e \(C=(1,-1,0)\)
            \item \(\pi\) contém \(A=(1,0,2)\), \(B=(-1,1,3)\) e \(C=(3,-1,1)\)
            \item \(\pi\) contém \(P=(1,0,-1)\) e \(r:(x-1)/2=y/3=2-z\)
            \item \(\pi\) contém \(P=(1,-1,1)\) e \(r:X=(0,2,2)+\lambda (1,1,-1)\)
        \end{enumerate}
    \end{minipage}
\end{frame}

\begin{frame}{Exercício 14-28}
    \begin{minipage}{\textwidth}
        Em uma ensolarada manhã de domingo, Marcelo resolveu dois exercícios que
        pediam equações gerais de planos. Em cada um dos itens seguintes, a primeira equação
        é a resposta que Marcelo obteve, e a segunda, a resposta do livro. Quais exercícios
        ele acertou?
    \end{minipage}
    \begin{enumerate}[(i)]
        \item \(\color{blue} x-3y+2z+1=0; ~ \color{red} 2x-6y+4z+4=0\)
        \item \(\color{blue} x-y/2+2z-1=0; ~ \color{red} -2x+y-4z+2=0\)
    \end{enumerate}

    \vspace {1.5cm}
    \pause

    \begin{tcolorbox}[colback=magenta!25]
    É de se esperar que existam infinitas equações gerais para um mesmo plano (ver Exercício Resolvido 14-20),
    mas há alguma restrição para os valores das constantes \(a\), \(b\), \(c\) e \(d\)?

    % Lembrando que, sendo \((x,y,z)=(x_0,y_0,z_0)+\lambda (r,s,t) + \mu (m,n,p)\), temos
    % \[
    %     \begin{vmatrix}
    %         s & t \\ n & p
    %     \end{vmatrix}
    %     =a \quad
    %     \begin{vmatrix}
    %         r & t \\ m & p
    %     \end{vmatrix}
    %     =b \quad
    %     \begin{vmatrix}
    %         r & s \\ m & n
    %     \end{vmatrix}
    %     =c \quad
    %     -ax_0- by_0 -cz_0 = d
    % \]
    \end{tcolorbox}

\end{frame}

\begin{frame}{Atividade}
    Resolva os exercícios 14-3, 14-23 e 14-30
\end{frame}

\begin{frame}[c]{Distância entre pontos, retas e planos}
    \begin{itemize}
        \item Se \(M\) e \(P\) são pontos, retas ou planos, a distância \(d(M,P)\) é
            o menor dos números \(d(X,Y)\), onde \(X\) ''pertence'' a \(M\) e 
            \(Y\) ''pertence'' a \(P\)
        \item Se há pontos comuns entre \(M\) e \(P\), temos que \(d(M,P)=0\)
    \end{itemize}
\end{frame}

\begin{frame}{Distância entre pontos, retas e planos}
    \begin{tcolorbox}[colback=magenta!10]
        Para obtermos a distância entre os pontos \(A=(x_1,y_1,z_1)\) e \(B=(x_2,y_2,z_2)\), 
        calculamos
        \[
            d(A,B)=\sqrt{(x_2-x_1)^2+(y_2-y_1)^2+(z_2-z_1) ^2}
        \]
    \end{tcolorbox}
    \pause
    \begin{tcolorbox}[colback=blue!10]
        Para obtermos a distância entre um ponto \(P\) e uma reta \(r\), primeiro
        encontramos dois pontos, \(A\) e \(B\), pertencentes a \(r\) e calculamos:
        \[
            d(P,r) = \frac{||\vec{AP}\wedge \vec{AB}||}{||\vec{AB}||}
        \]
    \end{tcolorbox}
    \pause
    \begin{tcolorbox}[colback=green!10]
        Para obtermos a distância entrem um ponto \(P=(x_0,y_0,z_0)\) e um 
        plano \(\pi: ax+by+cz+d=0\) usamos a fórmula:
        \[
        \displaystyle d(P,\pi) = \frac{|ax_0+by_0+cz_0+d|}{\sqrt{a^2+b^2+c^2}}
    \]
    \end{tcolorbox}
    \vspace{-3pt} %gambiarra
\end{frame}

\begin{frame}{Atividade}
    Exercícios 20-8, 20-10 e 20-21
\end{frame}
