\documentclass[brazilian, fleqn]{article}

\usepackage{babel}
\usepackage[utf8]{inputenc}
\usepackage[T1]{fontenc}
\usepackage{lmodern}

\usepackage{amssymb,amsfonts,amsmath}

\usepackage{tikz}
\usetikzlibrary{calc,intersections}

\DeclareMathOperator{\sen}{sen}
\DeclareMathOperator{\tg}{tg}
\renewcommand{\vec}[1]{\overrightarrow{#1}}

\usepackage[left=2cm, bottom=2cm, right=1.5cm, top=1.5cm]{geometry}

\usepackage{siunitx}
\sisetup{locale = FR}

\usepackage{tcolorbox}
\tcbuselibrary{skins}
\tcbset{boxrule=0pt, top=0pt, bottom=0pt, skin=bicolor, interior style={left color=black!10}}

\usepackage{enumitem}

\begin{document}
\section*{\centering Gabarito da prova ímpar}

\begin{enumerate}
    \item Os vetores não-nulos \(\vec{u}\) e \(\vec{v}\) são ortogonais, sendo \(||\vec{u}||=\tg{(\SI{60}{\degree})} ||\vec{v}||\), e
        \(\vec{w}\) é gerado por eles. Sabendo que \(\vec{w} \cdot \vec{u} = \vec{w} \cdot \vec{v} \) e que \(\vec{w}\)
        não é nulo, obtenha as medidas angulares, em graus, entre \(\vec{u}\) e \(\vec{w}\) e entre \(\vec{v}\) e
        \(\vec{w}\)

        \begin{tcolorbox}

            \begin{minipage}{0.45\textwidth}
                \begin{itemize}
                    \item Temos que
                        \[
                            \tg{\theta} = \frac{|\beta| ||\vec{v}||}{|\alpha| ||\vec{u}||}
                        \]

                    \item Considerando \(\phi\) a medida angular entre \(\vec{u}\) e \(\vec{w}\) e
                        \(\gamma\) a medida angular entre \(\vec{v}\) e \(\vec{w}\), temos as seguintes possibilidades:
                        \begin{itemize}[label={$\rightarrow$}]
                            \item \(\alpha > 0\): \(\phi = \theta\)
                            \item \(\alpha < 0\): \(\phi = \SI{180}{\degree} - \theta\)
                            \item \(\beta > 0\): \(\gamma = \SI{90}{\degree}-\theta\)
                            \item \(\beta < 0\): \(\gamma = \SI{90}{\degree} + \theta\)
                        \end{itemize}
                \end{itemize}
            \end{minipage}
            %
            \begin{minipage}{0.45\textwidth}
                \begin{center}
                    \begin{tikzpicture}
                        \draw [blue,->] (0,0) -- (5,0) node [midway, below] {$\alpha \vec{u}$};
                        \draw [red,->] (0,0) -- (0,4) node [midway, left] {$\beta \vec{v}$};
                        \draw [dashed,red,->] (5,0) -- (5,4);
                        \draw [green!50!black,->] (0,0) -- (5,4) node [midway, sloped, above] {$\vec{w}=\alpha \vec{u}+ \beta \vec{v}$};
                        \draw (0.5,0) arc (0:atan2(4,5):0.5) node [midway, right] {$\theta$};
                    \end{tikzpicture}
                \end{center}
            \end{minipage}

            \begin{itemize}
                \item Além disso, como \(\vec{u}\) e \(\vec{v}\) são ortogonais, temos que
                    \(\vec{w} \cdot \vec{u} = \alpha \vec{u} \cdot \vec{u} = \alpha ||\vec{u}||^2\) e
                    \(\vec{w} \cdot \vec{v} = \beta \vec{v} \cdot \vec{v} = \beta ||\vec{v}||^2\)
                \item Ou seja:
                    \[
                        \frac{|\vec{w} \cdot \vec{v}|}{|\vec{w} \cdot \vec{u}|}=
                        \frac{|\beta|||\vec{v}||^2}{|\alpha| ||\vec{u}||^2}=
                        \frac{||\vec{v}||}{||\vec{u}||}\tg {\theta} \implies
                        \tg{\theta} =
                        \frac{|\vec{w} \cdot \vec{v}|}{|\vec{w} \cdot \vec{u}|}
                        \frac{||\vec{u}||}{||\vec{v}||}
                    \]
                \item Do enunciado temos que \(\vec{w} \cdot \vec{u} = \vec{w} \cdot \vec{v}\),
                    que implica que \(\alpha\) e \(\beta\) têm o mesmo sinal (ambos positivos ou ambos negativos)
                    e que
                    \[
                        \tg {\theta} = \tg{\SI{60}{\degree}} \implies \theta = \SI{60}{\degree}
                    \]
                    onde usamos o fato de que \(0 < \theta < \SI{90}{\degree}\), decorrente de que \(\alpha\) e \(\beta\) não são nulos
                \item Assim, finalmente, temos que \(\phi=\SI{60}{\degree}\) e \(\gamma=\SI{30}{\degree}\) ou
                    \(\phi=\SI{120}{\degree}\) e \(\gamma=\SI{150}{\degree}\)

            \end{itemize}
        \end{tcolorbox}

    \item Sejam \(\vec{u}=(0,0,1)_B\) e \(\vec{v}=(1,2,0)_B\), onde \(B\) é uma base ortonormal positiva. Obtenha uma base
        ortonormal positiva \((\vec{a}, \vec{b}, \vec{c})\) tal que:
        \begin{itemize}
            \item \(\vec{a}\) e \(\vec{u}\) sejam de mesmo sentido;
            \item \(\vec{b}\) seja combinação linear de \(\vec{u}\), \(\vec{v}\);
            \item a primeira coordenada de \(\vec{b}\) seja positiva.
        \end{itemize}

        \begin{tcolorbox}
            \begin{itemize}
                \item Como \(\vec{a}=\alpha \vec{u}=(0,0,\alpha)_B\) e \(\vec{a}\) é unitário, temos que
                    \(\alpha = 1 \implies \vec{a}=(0,0,1)_B\)
                \item Temos que \(\vec{b}=\beta \vec{u} + \gamma \vec{v} = (\gamma, 2\gamma, \beta)_B\)
                \item Como queremos uma base ortonormal, temos que \(\vec{a}\cdot \vec{b} = 0 \implies \beta = 0\)
                \item Como \(\vec{b}\) é unitário, temos que \(\sqrt{\gamma^2+4\gamma^2}=1 \implies \gamma = \pm 1/\sqrt{5}\)
                \item Como a primeira coordenada de \(\vec{b}\) é positiva, temos que \(\vec{b} = (1,2,0)_B /\sqrt{5}\)
                \item Temos que \(\vec{c} = \vec{a} \wedge \vec{b}\), ou seja
                    \[
                        \vec{c} =
                        \begin{vmatrix}
                            \vec{i} & \vec{j} & \vec{k} \\
                            0 & 0 & 1 \\
                            1/\sqrt{5} & 2/\sqrt{5} & 0
                        \end{vmatrix}= 1/\sqrt{5} \vec{j}-2/\sqrt{5} \vec{i} \implies \vec{c} = (-2, 1, 0)_B /\sqrt{5}
                    \]
                    onde consideramos \(B=(\vec{i},\vec{j},\vec{k})\)
            \end{itemize}
        \end{tcolorbox}

    \item Em relação a um sistema ortogonal de coordenadas, \(A=(1,2,-1)\), \(B=(0,1,-2)\) e \(C=(2,0,0)\).
        Verifique se \(A\), \(B\) e \(C\) são vértices de um triângulo retângulo

        \begin{tcolorbox}
            \begin{itemize}
                \item Temos que \(\vec{AB}=(-1,-1,-1)\) e \(\vec{AC}=(1,-2,1)\)
                \item Como não existe \(\alpha\) tal que \(\vec{AB} = \alpha \vec{AC}\),
                    pois implicaria que \(\alpha = 2\alpha\) com \(\alpha \neq 0\), os
                    pontos \(A\), \(B\) e \(C\) não são colineares e portanto podem ser
                    vértices de um triângulo
                \item O teorema de Pitágoras diz que \(\text{(hipotenusa)}^2=\text{(cateto 1)}^2+\text{(cateto 2)}^2\). Como
                    \begin{align*}
                        ||\vec{AB}|| &=\sqrt{1+1+1}=\sqrt{3} \\
                        ||\vec{AC}|| &=\sqrt{1+4+1}=\sqrt{6} \\
                        ||\vec{BC}|| &=\sqrt{(2-0)^2+(0-1)^2+(0+2)^2}=3
                    \end{align*}
                    podemos verificar que \(||\vec{BC}||^2=||\vec{AB}||^2+||\vec{AC}||^2\) e confirmar que \(A\), \(B\) e
                    \(C\) podem ser vértices de um triângulo retângulo
            \end{itemize}
        \end{tcolorbox}
\end{enumerate}
\end{document}
