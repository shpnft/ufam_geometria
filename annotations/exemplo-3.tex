\documentclass[brazilian, fleqn]{article}

\usepackage{babel}
\usepackage[utf8]{inputenc}
\usepackage[T1]{fontenc}
\usepackage{lmodern}

\usepackage{amssymb,amsfonts,amsmath}

\usepackage{tikz}
\usetikzlibrary{tikzmark}
\usetikzlibrary{calc}

\DeclareMathOperator{\sen}{sen}

\usepackage[left=2cm, bottom=2cm, right=1.5cm, top=1.5cm]{geometry}

%https://tex.stackexchange.com/a/224014
\setlength{\jot}{1em}

\renewcommand{\vec}[1]{\overrightarrow{#1}}

\allowdisplaybreaks[1]

\begin{document}

\section{Resolução do exemplo 3a}

\begin{gather}
    %%https://tex.stackexchange.com/a/101760
    %\begin{tikzpicture}[remember picture, overlay]
    %    \foreach \i/\j in {f1/red,f2/blue,f3/green!50!black} {
    %    \draw [\j] ([yshift=2em] pic cs:\i a) rectangle ([yshift=-1em] pic cs:\i b);
    %    \draw [\j,->] (pic cs:\i b) to[out=0, in=90] ([yshift=2em] $(pic cs:f4a)!0.75!(pic cs:f4b)$);
    %}
    %    \draw [magenta] ([yshift=2em] pic cs:f4a) rectangle ([yshift=-1em] pic cs:f4b);
    %\end{tikzpicture}
    \vec{BG}= \vec{BM}+\vec{MG}=\tikzmark{f1a} \frac{\vec{BD}}{2}+\vec{MG} \tikzmark{f1b} \\
    \begin{aligned}[t]
        \vec{MN}&=\vec{MD}+\vec{DN} =\frac{\vec{BD}}{2}+\frac{\vec{DC}}{2} \\
                &=\frac{\vec{BD}}{2}-\frac{\vec{BD}}{2}+\frac{\vec{BC}}{2} \\
                &=\tikzmark{f2a} \frac{\vec{BC}}{2} \tikzmark{f2b}
    \end{aligned} \\
    \begin{aligned}[t]
        \vec{MP}&=\vec{MB}+\vec{BC}+\vec{CP}=-\frac{\vec{BD}}{2}+\vec{BC}+\frac{\vec{CA}}{2} \\
                &=-\frac{\vec{BD}}{2}+\vec{BC}-\frac{\vec{BC}}{2}+\frac{\vec{BA}}{2} \\
                &=\tikzmark{f3a} \frac{\vec{BA}}{2}+\frac{\vec{BC}}{2}-\frac{\vec{BD}}{2} \tikzmark{f3b}
    \end{aligned} \\
    \begin{aligned}[t]
        \vec{BG}&=\tikzmark{f4a} \frac{\vec{BD}}{2}+\alpha \frac{\vec{BC}}{2}+\beta \frac{\vec{BA}}{2}+\beta \frac{\vec{BC}}{2}-\beta \frac{\vec{BD}}{2} \tikzmark{f4b} \\
                &=\frac{\beta}{2}\vec{BA}+\frac{1}{2}(\alpha+\beta)\vec{BC}+\frac{1}{2}(1-\beta)\vec{BD}
    \end{aligned} \\
    \vec{MG}=\alpha \vec{MN}+\beta \vec{MP} \\
    \vec{MG}=\lambda \vec{MX} \\
    \begin{aligned}[t]
        \vec{MX}&=\vec{MN}+\vec{NX}=\vec{MN}+\frac{NP}{2}\\
                &=\vec{MN}-\frac{\vec{MN}}{2}+\frac{\vec{MP}}{2} \\
                &=\frac{\vec{MN}}{2}+\frac{\vec{MP}}{2}
    \end{aligned} \\
    \vec{MG}=\frac{\lambda}{2}(\vec{MN}+\vec{MP})\\
    \vec{NG}=\sigma (\vec{NM}+\vec{NP}) \\
    \vec{NG}=-\vec{MN}+\vec{MG}\\
    \vec{NP}=-\vec{MN}+\vec{MP}\\
    \vec{MG}-\vec{MN}=\frac{\sigma}{2}(-\vec{MN}-\vec{MN}+\vec{MP})\\
    \vec{MG}=(1-\sigma)\vec{MN}+\frac{\sigma}{2}\vec{MP} \\
    \begin{cases}
        1-\sigma &= \frac{\lambda}{2}\\
        \frac{\sigma}{2}&=\frac{\lambda}{2}
    \end{cases} \implies
    \lambda = \sigma = \frac{2}{3}\\
    \alpha = \beta = \frac{1}{3} \\
    \tikzmark{g1a} \vec{BG}=\frac{\vec{BA}}{6}+\frac{\vec{BC}}{3}+\frac{\vec{BD}}{3} \tikzmark{g1b}
    %%https://tex.stackexchange.com/a/101760
    %\begin{tikzpicture}[remember picture, overlay]
    %    \draw [magenta] ([yshift=2em] pic cs:g1a) rectangle ([yshift=-1em] pic cs:g1b);
    %\end{tikzpicture}
\end{gather}

\section{Resolução do exemplo 3b}
\begin{gather}
\vec{BX}=\frac{m}{6}\vec{BA}+\frac{m}{3}\vec{BC}+\frac{m}{3}\vec{BD}\\
\alpha \vec{DX}+\beta \vec{AC}+\gamma \vec{CD} = \vec{0} \\
\begin{aligned}[t]
    \vec{DX}&=\vec{DB}+\vec{BX}\\
            &=-\vec{BD}+\vec{BX}\\
            &=-\vec{BD}+\frac{m}{6}\vec{BA}+\frac{m}{3}\vec{BC}+\frac{m}{3}\vec{BD}\\
            &=\frac{m}{6}\vec{BA}+\frac{m}{3}\vec{BC}+\left(\frac{m}{3}-1\right)\vec{BD}
\end{aligned} \\
\vec{AC}=\vec{AB}+\vec{BC}=-\vec{BA}+\vec{BC}\\
\vec{CD}=\vec{CB}+\vec{BD}=-\vec{BC}+\vec{BD}\\
\alpha \left[\frac{m}{6}\vec{BA}+\frac{m}{3}\vec{BC}+\left(\frac{m}{3}-1\right)\vec{BD}\right]+
\beta \left(-\vec{BA}+\vec{BC}\right)+ \gamma \left(-\vec{BC}+\vec{BD}\right) = \vec{0} \\
\left(\alpha\frac{m}{6}-\beta\right)\vec{BA}+\left(\alpha\frac{m}{3}+\beta-\gamma\right)\vec{BC}+
\left[\alpha\left(\frac{m}{3}-1\right)+\gamma\right]\vec{BD}=\vec{0} \\
\begin{cases}
    \alpha\frac{m}{6}-\beta &=0 \\
    \alpha\frac{m}{3}+\beta-\gamma &=0 \\
    \alpha\left(\frac{m}{3}-1\right)+\gamma &=0
\end{cases} \\
\begin{vmatrix}
    \frac{m}{6} & -1 & 0 \\
    \frac{m}{3} & 1 & -1 \\
    \frac{m}{3}-1 & 0 & 1
\end{vmatrix} = 0 \\
\frac{m}{6}+\frac{m}{3}-1+\frac{m}{3}=0 \\
m=\frac{6}{5}
\end{gather}

\end{document}
