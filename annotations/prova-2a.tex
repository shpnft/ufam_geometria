\documentclass[12pt,a4paper,brazilian, fleqn]{article}

\usepackage{babel}
\usepackage[utf8]{inputenc}
\usepackage[T1]{fontenc}
\usepackage{lmodern}

\usepackage{amssymb,amsfonts,amsmath}

\usepackage{tikz}
\usetikzlibrary{calc,intersections}

\usepackage{tcolorbox}
\tcbset{boxrule=0pt, top=0pt, bottom=0pt}

\DeclareMathOperator{\sen}{sen}
\DeclareMathOperator{\tg}{tg}

%https://tex.stackexchange.com/a/100406
%29.7 cm - 1cm - 1cm - 144.90/28.4 cm = 22.60 cm
\usepackage[a4paper, totalheight=22.60cm,includeheadfoot,left=1.5cm, right=1.0cm, top=1cm]{geometry}
\setlength{\headheight}{144.90pt}

\setkeys{Gin}{keepaspectratio}

\newcommand{\cabeca}{
    \begin{tikzpicture}
        \node(Logo) {\includegraphics[width=2.5cm]{logo.png}};
        % \node(Logo) {\includegraphics[width=4.1cm]{logo.png}};

        \node(Local) at (Logo.north east) [anchor=north west, yshift=-0.25cm,
            align=center, execute at begin node=\setlength{\baselineskip}{3ex}]
            {
                \huge{\textbf{Universidade Federal do Amazonas}} \\
                \large{\textbf{Instituto de Ciências Exatas e Tecnologia}} \\
                \large{\textbf{\Description}}
            };

        \node(Ident) at (Local.south west) [anchor=north west, yshift=-0.25cm,
            align=left, execute at begin node=\setlength{\baselineskip}{2em}]
            {
                Professor: {\fontfamily{augie}\selectfont \Professor} \\
                Aluno:
            };
        % \draw [thick] (Logo.south west) -- ($(Logo.south west -| Local.south east)$);
        % \draw [red] (Logo.north west) rectangle (Logo.south east);
        % \draw [blue] (Local.north west) rectangle (Local.south east);
        % \draw [green] (Ident.north west) rectangle (Ident.south east);
    \end{tikzpicture}
}

\usepackage{fancyhdr}
\fancyhead{}
\fancyfoot{}
\fancyhead[c]{\cabeca}
\fancyfoot[r]{\fontfamily{augie}\selectfont Boa sorte!}


\pagestyle{fancy}
\renewcommand{\headrulewidth}{0pt}
\renewcommand{\footrulewidth}{0pt}

\newcommand{\ratio}[1]{(#1\% da nota)}
\renewcommand*{\thefootnote}{\fnsymbol{footnote}}
%-----------------------------------CUT HERE-----------------------------------

\def\Description{Geometria Analítica -- Prova 2}
\def\Professor{Rodrigo de Farias Gomes}

\renewcommand{\vec}[1]{\overrightarrow{#1}}

\usepackage{siunitx}
\sisetup{locale = FR}

\begin{document}

\begin{tcolorbox}[colback=black!10, colframe=black!50, title=Observações]
    \begin{itemize}
        \item Todas as páginas com resposta devem ter o nome e matrícula do aluno
            escritos com caneta no início (cabeçalho) ou no final (rodapé). Páginas
            que não obedeçam a esse critério não serão usadas na avaliação
        \item As respostas podem ser escritas com lápis desde que legível
    \end{itemize}
\end{tcolorbox}

\begin{enumerate}
    \item \ratio{34} Os vetores não-nulos \(\vec{u}\) e \(\vec{v}\) são ortogonais, sendo \(||\vec{u}||=\tg{(\SI{60}{\degree})} ||\vec{v}||\), e
        \(\vec{w}\) é gerado por eles. Sabendo que \(\vec{w} \cdot \vec{u} = \vec{w} \cdot \vec{v} \) e que \(\vec{w}\)
        não é nulo, obtenha as medidas angulares, em graus, entre \(\vec{u}\) e \(\vec{w}\) e entre \(\vec{v}\) e
        \(\vec{w}\)

    \item \ratio{33} Sejam \(\vec{u}=(0,0,1)_B\) e \(\vec{v}=(1,2,0)_B\), onde \(B\) é uma base ortonormal positiva. Obtenha uma base
        ortonormal positiva \((\vec{a}, \vec{b}, \vec{c})\) tal que:

        \begin{itemize}
            \item \(\vec{a}\) e \(\vec{u}\) sejam de mesmo sentido;
            \item \(\vec{b}\) seja combinação linear de \(\vec{u}\), \(\vec{v}\);
            \item a primeira coordenada de \(\vec{b}\) seja positiva.
        \end{itemize}

    \item \ratio{34} Em relação a um sistema ortogonal de coordenadas, \(A=(1,2,-1)\), \(B=(0,1,-2)\) e \(C=(2,0,0)\).
        Verifique se \(A\), \(B\) e \(C\) são vértices de um triângulo retângulo
\end{enumerate}

\end{document}
