\documentclass[t, brazilian, 11pt, aspectratio=169]{beamer}

\usepackage{babel}
\usepackage[utf8]{inputenc}
\usepackage[T1]{fontenc}
\usepackage{lmodern}

\usepackage{amssymb,amsfonts,amsmath}

\usetheme{Boadilla}
\setbeamertemplate{navigation symbols}{}
\setbeamertemplate{frametitle continuation}{}
\setbeamertemplate{page number in head/foot}[framenumber]
\setbeamertemplate{enumerate items}[default]
\setbeamertemplate{itemize items}[circle]
\setbeamercovered{transparent}

\def\Disciplina{Geometria Analítica}
\def\Professor{Rodrigo de Farias Gomes}
\def\Periodo{Período 2022.1}

\title{\Disciplina}
\author{\Professor}
\date{\Periodo}

\renewcommand{\vec}[1]{\overrightarrow{#1}}

\begin{document}
\begin{frame}{Atividade 1}
    Atividade para a próxima aula:
    \begin{itemize}
        \item Vale 2 pontos extras na \(N_2\) (nota máxima 10) para todos os 
            presentes na próxima aula
        \item Um aluno presente na aula de hoje será sorteado na próxima aula
            e deve, na próxima aula:
            \begin{itemize}
                \item Explicar e dar um exemplo de um ângulo inscrito numa 
                    semicircunferência
                \item Resolver o exercício 9-30 do livro texto (página 78)
            \end{itemize}
        \item Ambos devem ser no quadro
        \item Perguntas do professor sobre a explicação ou resolução devem ser
            respondidas
        \item Se o aluno escolhido não conseguir fazer a atividade:
            \begin{itemize}
                \item os alunos que faltarem na próxima aula perderão 2 pontos na \(N_2\) 
                    (nota mínima 0), caso não tenham justificativa
                \item Os alunos que estiverem na próxima aula terão que resolver e 
                    entregar manuscrito (completo, não é só a resposta) os exercícios
                    9-1 até 9-50 do livro texto ou perderão 2 pontos na \(N_2\) (nota mínima 0)
            \end{itemize}
    \end{itemize}
\end{frame}

\begin{frame}{Lista de alunos}
    \begin{columns}[c]
        \begin{column}{0.4\textwidth}
            \begin{enumerate}
                \item Emanuel
                \item Horanna
                \item Leonardo
                \item Mayara
                \item Nalbert
                \item Pedro
            \end{enumerate}
        \end{column}
        \begin{column}{0.4\textwidth}
            \centering
            \includegraphics[width=0.8\textwidth]{../images/dice.jpg}
        \end{column}
    \end{columns}
\end{frame}
\end{document}
