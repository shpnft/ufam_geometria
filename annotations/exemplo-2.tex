\documentclass[brazilian, fleqn]{article}

\usepackage{babel}
\usepackage[utf8]{inputenc}
\usepackage[T1]{fontenc}
\usepackage{lmodern}

\usepackage{amssymb,amsfonts,amsmath}

\DeclareMathOperator{\sen}{sen}

\usepackage[left=2cm, bottom=2cm, right=1.5cm, top=1.5cm]{geometry}

%https://tex.stackexchange.com/a/224014
\setlength{\jot}{1em}

\renewcommand{\vec}[1]{\overrightarrow{#1}}

\begin{document}

\section{Resolução do exemplo 2}

\begin{gather}
    \vec{AX} = \frac{m}{3}\vec{AD} - m \vec{AB} +\frac{m}{2} \vec{AC} \\
    \alpha \vec{DX} +\beta \vec{BC} + \gamma \vec{CD} = \vec{0} \\
    % https://tex.stackexchange.com/a/4424
    \begin{aligned}[t]
        \vec{DX} &=\vec{DA}+\vec{AX} \\
                 &=-\vec{AD}+\vec{AX}\\
                 &=-\vec{AD}+\frac{m}{3}\vec{AD} - m \vec{AB} +\frac{m}{2} \vec{AC} \\
                 &= \left(\frac{m}{3}-1\right)\vec{AD}- m \vec{AB} +\frac{m}{2} \vec{AC} 
    \end{aligned} \\
    %
    \begin{aligned}[t]
        \vec{BC}&=\vec{BA}+\vec{AC}\\
                &= -\vec{AB}+\vec{AC} 
    \end{aligned} \\
    %
    \begin{aligned}[t]
        \vec{CD}&=\vec{CA}+\vec{AD}\\
                &= -\vec{AC}+\vec{AD} 
    \end{aligned} \\
    %
    \alpha \left[ \left(\frac{m}{3}-1\right)\vec{AD}- m \vec{AB} +\frac{m}{2} \vec{AC} \right]
    +\beta \left( -\vec{AB}+\vec{AC} \right)+\gamma\left( -\vec{AC}+\vec{AD} \right)=\vec{0} \\
    %
    \left[\alpha \left(\frac{m}{3}-1\right) +\gamma\right] \vec{AD}+\left(-\alpha m-\beta\right)\vec{AB}+
    \left(\alpha\frac{m}{2} +\beta-\gamma\right)\vec{AC}=\vec{0}\\
    %
    \begin{cases}
        \alpha \left(\frac{m}{3}-1\right) +\gamma = 0 \\
        -\alpha m-\beta=0 \\
        \alpha\frac{m}{2} +\beta-\gamma = 0
    \end{cases}
    \implies
    \begin{cases}
        \alpha \left(\frac{m}{3}-1\right) +\gamma = 0 \\
        \alpha m+\beta=0 \\
        \alpha\frac{m}{2} +\beta-\gamma = 0
    \end{cases} \\
    \begin{vmatrix}
        \frac{m}{3}-1 & 0 & 1 \\
        m & 1  & 0 \\
        \frac{m}{2} & 1 & -1
    \end{vmatrix} =0 \\
    -\left(\frac{m}{3}-1\right)+m+0-\left(\frac{m}{2}+0+0\right)=-\frac{m}{3}+1+m-\frac{m}{2}=0 \\
    m = -6
\end{gather}
\end{document}
