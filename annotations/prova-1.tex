\documentclass[12pt,a4paper,brazilian, fleqn]{article}

\usepackage{babel}
\usepackage[utf8]{inputenc}
\usepackage[T1]{fontenc}
\usepackage{lmodern}

\usepackage{amssymb,amsfonts,amsmath}

\usepackage{tikz}
\usetikzlibrary{calc,intersections}

\usepackage{tcolorbox}
\tcbset{boxrule=0pt, top=0pt, bottom=0pt}

\DeclareMathOperator{\sen}{sen}

%https://tex.stackexchange.com/a/100406
%29.7 cm - 1cm - 1cm - 144.90/28.4 cm = 22.60 cm
\usepackage[a4paper, totalheight=22.60cm,includeheadfoot,left=1.5cm, right=1.0cm, top=1cm]{geometry}
\setlength{\headheight}{144.90pt}

\setkeys{Gin}{keepaspectratio}

\newcommand{\cabeca}{
    \begin{tikzpicture}
        \node(Logo) {\includegraphics[width=2.5cm]{logo.png}};
        % \node(Logo) {\includegraphics[width=4.1cm]{logo.png}};

        \node(Local) at (Logo.north east) [anchor=north west, yshift=-0.25cm,
            align=center, execute at begin node=\setlength{\baselineskip}{3ex}]
            {
                \huge{\textbf{Universidade Federal do Amazonas}} \\
                \large{\textbf{Instituto de Ciências Exatas e Tecnologia}} \\
                \large{\textbf{\Description}}
            };

        \node(Ident) at (Local.south west) [anchor=north west, yshift=-0.25cm,
            align=left, execute at begin node=\setlength{\baselineskip}{2em}]
            {
                Professor: {\fontfamily{augie}\selectfont \Professor} \\
                Aluno:
            };
        % \draw [thick] (Logo.south west) -- ($(Logo.south west -| Local.south east)$);
        % \draw [red] (Logo.north west) rectangle (Logo.south east);
        % \draw [blue] (Local.north west) rectangle (Local.south east);
        % \draw [green] (Ident.north west) rectangle (Ident.south east);
    \end{tikzpicture}
}

\usepackage{fancyhdr}
\fancyhead{}
\fancyfoot{}
\fancyhead[c]{\cabeca}
\fancyfoot[r]{\fontfamily{augie}\selectfont Boa sorte!}


\pagestyle{fancy}
\renewcommand{\headrulewidth}{0pt}
\renewcommand{\footrulewidth}{0pt}

\newcommand{\ratio}[1]{(#1\% da nota)}
%-----------------------------------CUT HERE-----------------------------------

\def\Description{Geometria Analítica -- Prova 1 16/12/2022}
\def\Professor{Rodrigo de Farias Gomes}

\renewcommand{\vec}[1]{\overrightarrow{#1}}

\begin{document}

\begin{tcolorbox}[colback=black!10, colframe=black!50, title=Observações]
    \begin{itemize}
        \item Todas as páginas com resposta devem ter o nome e matrícula do aluno
            escritos com caneta no início (cabeçalho) ou no final (rodapé). Páginas
            que não obedeçam a esse critério não serão usadas na avaliação
        \item As respostas podem ser escritas com lápis desde que legível
    \end{itemize}
\end{tcolorbox}

\begin{enumerate}
    \item \ratio{30} Determine \(\vec{x}\) com o auxílio da Figura 1
        \begin{enumerate}
            \item \(\vec{x}=\vec{HE}-\vec{BD}\)
            \item \(\vec{x}=\vec{AB}+\vec{EH}\)
            \item \(\vec{x}=\vec{AF}+\vec{GH}\)
        \end{enumerate}
    \item \ratio{40} No tetraedro \(ABCD\) (Figura 2), sejam \(M\), \(N\) e \(P\), respectivamente,
        os pontos médios de \(BC\), \(CD\) e \(AC\), e \(G\) o baricentro do triângulo \(MNP\).
        \begin{enumerate}
            \item Mostre que \(\vec{CG}=\dfrac{1}{6}(\vec{CA}+\vec{CB}+\vec{CD})\)
            \item Mostre que \(m=2\) para que o ponto \(H=C+m\vec{CG}\) pertença ao plano da face \(ABD\)
            \item (2 pontos extras) Mostre que o ponto \(H\) do item anterior é baricentro do triângulo \(ABD\)
        \end{enumerate}

    \item \ratio{30} Em cada caso, calcule \(m\) para que os vetores sejam LD
        \begin{enumerate}
            \item \(\vec{u}=(m,1,m),~\vec{v}=(1,m,1)\)
            \item \(\vec{u}=(1-m^2,1-m,0),~\vec{v}=(m,m,m)\)
            \item \(\vec{u}=(m,1,m+1),~\vec{v}=(1,2,m),~\vec{w}=(1,1,1)\)
            \item \(\vec{u}=(m,1,m+1),~\vec{v}=(0,1,m),~\vec{w}=(0,m,2m)\)
        \end{enumerate}
\end{enumerate}

\centering
\begin{tikzpicture}[>=stealth, scale=0.9]
    \coordinate (A) at (0.35,0.56);
    \coordinate (B) at (2.19,0.56);
    \coordinate (C) at (1.3,4.75);
    \coordinate (D) at (3.14,4.75);
    \coordinate (E) at (2.65,5.75);
    \coordinate (F) at (4.49,5.75);
    \coordinate (G) at (3.55,1.55);
    \coordinate (H) at (1.71,1.55);

    \foreach \a/\b in {A/B,A/C,C/D,C/E,E/F,F/D,G/F,B/G,B/D}{
        \draw [thick] (\a) -- (\b);
    }
    \foreach \b in {G,A,E} {
        \draw [thick, dotted] (H) -- (\b);
    }

    \node [below left]  at (A) {A};
    \node [below right] at (B) {B};
    \foreach \a in {C,...,H} \node [above left] at (\a) {\a};

    \path (A) -- (A -| F) node [yshift=-1cm,shape=rectangle,fill=black!5,midway] {Figura 1};
\end{tikzpicture}
%
\hspace{3cm}%
%
\begin{tikzpicture}[scale=1.6*0.9]
    \coordinate (A) at (2.35, 3.6);
    \coordinate (B) at (2.75,0.4);
    \coordinate (C) at (3.8,1.09);
    \coordinate (D) at (0.6,0.95);
    \node [above]  at (A) {A};
    \node [below]  at (B) {B};
    \node [right]  at (C) {C};
    \node [left]   at (D) {D};

    \draw [thick, fill=gray!20] (A) -- (B) -- (D) -- cycle;
    \draw [thick] (B) -- (C) -- (A);

    \draw [thick, dotted] (C) -- (D);

    \path (B -| D) -- (B -| C) node [yshift=-1cm,shape=rectangle,fill=black!5,midway] {Figura 2};
\end{tikzpicture}

\end{document}
